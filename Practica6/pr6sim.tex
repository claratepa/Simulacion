\documentclass{article}
\usepackage[spanish]{babel}
\usepackage[T1]{fontenc}
\usepackage[ansinew]{inputenc}
\usepackage{graphicx}
\usepackage{url}
\usepackage[maxbibnames=99, sorting=none, backend=bibtex]{biblatex}
\addbibresource{bibpr2sim.bib}

\begin{document}
\title{\textbf{Sistema Multiagente}}
\author{Clara T\'ellez}
\maketitle

\section{Objetivo}\label{obj}

 La pr\'actica consite en estudiar el efecto de la variaci\'on en la probabilidad de agentes vacunados (de cero a uno, en pasos de 0.1) en el inicio de la rutina sobre el porcentaje m\'aximo de los agentes infectados durante la simulaci\'on \cite{eli}.

\section{Metodolog\'ia}\label{met}

Para determinar el efecto de la probabilidad de vacunaci\'on sobre los agentes infectados y su respectivo tratamiento estad\'istico se us\'o R en su versi\'on 3.6.2.


La rutina se dise�\'o variando la probabilidad de vacunaci\'on entre cero y uno en pasos de 0.1, se realizaron 25 repeticiones de cada probabilidad y se calcul\'o el m\'aximo n\'umero de infectados en cada simulaci\'on.  Con los datos obtenidos se realiz\'o una gr\'afica de cajas y bigotes y se hizo la prueba estad\'istica ANOVA.  



\section{Resultados y Discusi\'on}\label{res}

La probabilidad de vacunaci\'on inicial tiene un efecto directo sobre el n\'umero m\'aximo de infectados en la simulaci\'on.  A medida que aumenta la probabilidad de vacunaci\'on inicial, disminuye la cantidad maxima de infectados.  En la figura \ref{f1} se puede observar f\'acilmente el comportamiento del n\'umero m\'aximo de infectados en relaci\'on con la probabilidad de agentes vacunados.


\begin{figure}
  \begin{center}
    \includegraphics[width=10cm]{Pr6sim.png}
  \end{center}
  \caption{Porcentaje m\'aximo de infectados }
  \label{f1}
\end{figure}

Aunque al graficar los datos se pueden apreciar las diferencias entre las probabilides de vacunaci\'on, se realizo la prueba estad\'istica ANOVA para determinar si las diferencias son significativas (cuadro \ref{t1}).  El an\'alisis estad\'istico muestra  una significancia menor a 0.05 por lo que se rechaza la hip\'otesis nula y se puede afirmar que existen diferencias significativas en el n\'umero m\'aximo de infectados al variar la probabilidad de vacunaci\'on en una simulaci\'on de sistema multiagente.   
 

\begin{table} 
 \caption{Comparaci\'on de los porcentajes del m\'aximo de agentes infectados con respecto a la probabilidad de vacunaci\'on}
 \label{t1}
 \begin{center}
 \begin{tabular}{r|r|r|r|r|r}
\texttt{} & \texttt{GL} & \texttt{Suma Cuad.} &\texttt{Media Cuad.} & \texttt{F}  & \texttt{Pr(>F)} \\
\hline
datos-probabilidad & 1 & 60757 & 60757 & 239.7 & <2e-16\\ 
\hline
Residuales & 273& 69189 & 253 \\ 
\end{tabular}
\end{center}
\end{table}


\section{Conclusiones}\label{con}   

El aumento de la probabilidad de vacunaci\'on disminuye el porcentaje del n\'umero m\'aximo de infectados en una simulaci\'on de sistema multiagente.
\printbibliography
\end{document}