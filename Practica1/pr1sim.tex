\documentclass{article}
\usepackage[spanish]{babel}
\usepackage[T1]{fontenc}
\usepackage[ansinew]{inputenc}
\usepackage{graphicx}
\usepackage{float}

\begin{document}
\title{\textbf{Movimiento Browniano}}
\author{Clara T\'ellez}
\maketitle

\section{Introducci\'on}\label{intro}

El movimiento Browniano es un fen\'omeno f\'isico en el que una part\'icula suspendida en un fluido presenta movimiento continuo de naturaleza aleatoria.  El bot\'anico Robert Brown fu\'e el primero en observarlo en 1827, pero fue Albert Einstein quien lo explic\'o en 1905 \cite{Baz}.  

Al simular el movimiento Browniano se representan estos movimientos como si se tratara de una caminata, donde la part\'icula parte de un punto de origen y da ``pasos'' discretos (duraci\'on) de forma aleatoria y puede hacerlo en 1 o m\'as dimensiones \cite{Eli}.  A partir de este tipo de estudios se pueden predecir tendencias en el comportamiento de las part\'iculas de acuerdo a las variables que se tengan en cuenta. 

El objetivo de este trabajo es examinar los efectos de la dimensi\'on  en la probabilidad de regreso al origen, as\'i como, el efecto de la duraci\'on de la caminata en el comportamiento.

\section{Metodolog\'ia}\label{met}

Para evaluar los efectos de las dimensiones y de la duraci\'on de la caminata en la probabilidad de regreso al origen se us\'o el paquete estad\'istico R en su versi\'on 3.6.2.  

El experimento consisti\'o en simular una caminata variando las dimensiones entre 1 y 8 y tambi\'en los pasos de la misma  como potencias de dos con exponente de 5 a 10 en incrementos lineales de uno.  Se realizaron 50 repeticiones para cada caso, con estos datos se calcul\'o la probabilidad de regreso para cada una de las 8 dimensiones, as\'i mismo, se observ\'o el efecto de la duraci\'on de la caminata sobre el comportamiento de la part\'icula.

Los datos obtenidos fueron graficados en un diagrama de cajas y bigotes, en el que es posible visualizar, claramente, el efecto de las dimensiones en el comportamiento de una part\'icula que presenta movimiento Browniano.

\section{Resultados y Discusi\'on}\label{res}

Al ejecutar la simulaci\'on en R se obtuvo una matriz de datos correspondiente a los porcentajes de regreso al punto de origen para cada dimensi\'on y cada duraci\'on de la caminata.  En la figura 1 se muestra el conjunto de datos con el que se realizar\'a el an\'alisis de esta pr\'actica.

\graphicspath{ {c:/Users/User/Desktop/CITP/Simulacion/R/} }
\begin{figure}[h]
  \includegraphics[scale=0.65]{Fig1}
  \caption{Matriz de datos obtenida en R.  (pot=Potencia, porc=Porcentaje de regreso al punto de origen, dim=Dimensi\'on)}
  \label{Figura 1}
\end{figure}

A partir de la matriz de datos se construy\'o un diagrama de cajas de bigotes para observar el comportamiento de una part\'icula con movimiento Browniano (Figura 2).  Cada caja corresponde a una dimensi\'on en donde se agrupan los porcentajes de regreso al punto de origen en las diferentes duraciones de la caminata.


La dimensi\'on afecta la probabilidad de regreso al origen, a mayor dimensi\'on, menor probabilidad de regreso.  En la caja correspondiente a la dimensi\'on 1, los datos se distribuyen de manera uniforme, no hay datos at\'ipicos, ni se forman bigotes, la probabilidad de regreso esta muy cerca al 100\%.  La caja que representa la dimensi\'on 2 se aleja considerablemente de la caja de la dimensi\'on 1, los datos est\'an mas dispersos en especial en el nivel superior.  La distancia que hay entre las cajas 2 y 3 es m\'as amplia que la observada entre las cajas 1 y 2.  A partir de la caja 3 se va disminuyendo la distancia entre las cajas para las dimensiones m\'as grandes.  En la caja que corresponde a la dimensi\'on 4 podemos observar un dato at\'ipico en la parte inferior, es el \'unico dato at\'ipico obtenido para esta matriz, por lo que estadisticamente es no significativo.  
En el gr\'afico tambie\'en podemos inferir que la duraci\'on de la caminata no afecta la probabilidad de regreso al origen.  Las cajas, en general, son uniformes y peque\~nas, lo que evidencia que los datos se agrupan en rangos peque\~nos.

\graphicspath{ {c:/Users/User/Desktop/CITP/Simulacion/R/} }
\begin{figure}[H]
  \includegraphics[scale=0.7]{pr1si}
  \caption{Diagrama de cajas y bigotes}
  \label{Figura 2}
\end{figure}


\section{Conclusiones}\label{con}     
 
1.  La dimensi\'on afecta la probabilidad de regreso al origen.

2. La duraci\'on de la caminata no tiene efecto sobre la probabilidad de regreso al origen.


\begin{thebibliography}{0}
\bibitem{Baz} \textsc{Santamar\'ia-Antonio, J.},\textit{El movimiento Browniano: Un paradigma de la materia blanda y de la biolog\'ia}, Rev.R.Acad.Cienc.Exact.F\'is.Nat. (Esp),Vol. 106, Nº. 1-2, pp 39-54, 2013.
\bibitem{Eli} \textsc{Schaeffer, E.},\textit{Pr\'actica 1: Movimiento Browniano}, https://elisa.dyndns-web.com/teaching/comp/par/p1.html.  Consultada en enero de 2020.
\end{thebibliography}

\end{document}
