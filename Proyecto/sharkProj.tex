\documentclass[12pt, letterpaper] {article}

\parindent=5mm
\usepackage[spanish]{babel}

\usepackage{amssymb}
\usepackage{amsmath} 
\usepackage{amsfonts}

\usepackage[numbers,sort&compress]{natbib}
\usepackage{graphicx}

\usepackage{url}
\usepackage{hyperref}

\usepackage[top=25mm, bottom=20mm, left=1.5cm, right=1.5cm]{geometry}
\setlength{\parskip}{2mm}
\setlength{\parindent}{1pt}

\usepackage{listings}

\usepackage{float}

\usepackage[utf8]{inputenc}
\usepackage{graphicx} 
\usepackage{subfigure} 

\usepackage{color}
\usepackage{multirow}

\definecolor{dkgreen}{rgb}{0,0.6,0}
\definecolor{gray}{rgb}{0.5,0.5,0.5}
\definecolor{mauve}{rgb}{0.58,0,0.82}

\usepackage{color}
\usepackage{listings}
\lstset{ %
  language=R,                     % the language of the code
  basicstyle=\footnotesize,       % the size of the fonts that are used for the code
  numbers=left,                   % where to put the line-numbers
  numberstyle=\tiny\color{gray},  % the style that is used for the line-numbers
  stepnumber=1,                   % the step between two line-numbers. If it's 1, each line
                                  % will be numbered
  numbersep=5pt,                  % how far the line-numbers are from the code
  backgroundcolor=\color{white},  % choose the background color. You must add \usepackage{color}
  showspaces=false,               % show spaces adding particular underscores
  showstringspaces=false,         % underline spaces within strings
  showtabs=false,                 % show tabs within strings adding particular underscores
  frame=single,                   % adds a frame around the code
  rulecolor=\color{black},        % if not set, the frame-color may be changed on line-breaks within not-black text (e.g. commens (green here))
  tabsize=2,                      % sets default tabsize to 2 spaces
  captionpos=b,                   % sets the caption-position to bottom
  breaklines=true,                % sets automatic line breaking
  breakatwhitespace=false,        % sets if automatic breaks should only happen at whitespace
  title=\lstname,                 % show the filename of files included with \lstinputlisting;
                                  % also try caption instead of title
  keywordstyle=\color{blue},      % keyword style
  commentstyle=\color{dkgreen},   % comment style
  stringstyle=\color{mauve},      % string literal style
  escapeinside={\%*}{*)},         % if you want to add a comment within your code
  morekeywords={*,...}            % if you want to add more keywords to the set
} 


\author{Clara T\'ellez \\[0.9mm]}

\title{Estudio isot\'opico de la dieta del tibur\'on martillo}


\date{\today}

\begin{document}


\twocolumn[
\begin{@twocolumnfalse}

\maketitle

\begin{center}\rule{0.9\textwidth}{0.1mm} \end{center}
\begin{abstract}

\normalsize 

Aqui va el abstract\\ \\

Palabras clave: 
\begin{center}\rule{0.9\textwidth}{0.1mm} \end{center}
\end{abstract}

\end{@twocolumnfalse} 
]

\section{Introducción}
Los tiburones son unos de los principales depredadores de los oc\'eanos y por lo tanto desempe\~nan un papel esencial dentro de la red tr\'ofica, son ellos, quienes se encargan de mantener el equilibrio, aliment\'andose de las especies m\'as abundantes o de especies invasoras y favoreciendo la diversidad de las mismas.  Adicionalmente, de forma indirecta mantienen los arrecifes de coral \cite{Clarke}.   \textbf{\textit{Sphyrna lewini}} es una especie de tibur\'on martillo que habita en los mares tropicales y se encuentra tanto en aguas costeras como oce\'anicas \cite{Duncan}.  La uni\'on de la conservaci\'on de la naturaleza (IUCN por sus siglas en ingl\'es) considera al tibur\'on martillo como una especie cr\'iticamente amenazada debido a la disminuci\'on progresiva de sus poblaciones \cite{IUCN}.  

Estudiar  a detalle las diferentes caracter\'isticas de las especies es vital para establecer planes de manejo y conservaci\'on.  El estudio de los cambios ontogen\'eticos proporciona informaci\'on acerca del uso de nichos y del funcionamiento de los ecosistemas \cite{newsome}, sin embargo, la naturaleza migratoria y la inaccesibilidad del h\'abitat de los tiburones representan grandes dificultades para el estudio de estos cambios en la ecolog\'ia tr\'ofica del tibur\'on martillo \cite{hazen}.  Para establecer una verdadera red tr\'ofica es necesario entender las conexiones entre los organismos de la red y sus fuentes de alimento, as\'i como calificar y cuantificar los intercambios de energ\'ia y biomasa \cite {kling, vander}.  Los isot\'opos estables son una gran herramienta para el estudio, a profundidad, de las redes tr\'oficas, estos pueden permanecer sin combinarse con otros elementos, por lo que son muy \'utiles para el estudio de aspectos ecol\'ogicos, entre ellos, la dieta \cite{zanden}.   El an\'alisis de is\'otopos estables se realiza a partir de la composici\'on isot\'opica de un tejido, que refleja la presa asimilada y el ambiente de la misma y que permanece por largos per\'iodos de tiempo \cite{kim}.  Las vertebras, en particular, reflejan los cambios en la dieta de un organismo durante su vida \cite{koch}.  Este estudio pretende simular las interacciones tr\'oficas de \textbf{\textit{S.\ lewini}} a lo largo de su ontogenia alimentaria a trav\'es del estudio de is\'otopos estables obtenidos de las vertebras de diferentes individuos de la especie.







\section{Metodolog\'ia}

Se obtuvieron datos de is\'otopos de carbono y de nitr\'ogeno de 16 individuos de  \textbf{\textit{S.\ lewini}}. Los an\'alisis de is\'otopos estables se realizaron a partir de las vertebras de tiburones provenientes de la pesca ilegal, que fueron confiscados en los alrededores del Santuario de flora y fauna Malpelo, en  Colombia. Las vertebras fueron recolectadas de la parte dorso-anterior de cada esp\'ecimen, entre la cabeza y la primera aleta dorsal. Las vertebras fueron preparadas adecuadamente para la extracción de las muestras. En total, se procesaron 101 muestras, que fueron analizadas para obtener los datos isot\'opicos y la edad correspondiente.
La edad de los tiburones fue estimada a partir de las vertebras de acuerdo a la funci\'on de crecimiento de von Bertalanffy \cite{tolentino}. Debido a que los individuos eran provenientes de la pesca ilegal no fue posible establecer el sexo de todos los espec\'imenes, por lo que se calcul\'o la edad para hembra y  macho del mismo individuo y se promedi\'o.


 
\section{Resultados y discusión}


\section{Conclusiones}
  
\subsection*{Agradecimientos}


\bibliographystyle{unsrt}
\bibliography{refproj}




\end{document}